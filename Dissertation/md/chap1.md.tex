# Анализ концептуальных подходов, существующей практики, современного состояния и тенденций развития системы управления коммерческой недвижимостью в России\label{ch:ch1}
<!-- Главы можно писать по-разному. Я очень люблю стиль, в котором первая глава это постановка задачи. Часто в эту же главу входит обзор литературы. Обзор литературы можно раскидать по главам, но обычно кандидатские содержат обзор в первой главе. Я больше люблю, когда на каждую главу есть введение, заключение и обзор литературы. Все это индивидуально и зависит от вас и вашего научного руководителя. -->

## Систематизация и обобщение методов управления коммерческой недвижимостью в России\label{sec:ch1/sect1}
<!--
**Исходные данные:**
10+ международных и российских статистических и аналитических источников
**Содержание:**
Описание текущего состояния, трендов, прогнозов развития объекта исследования. Выявление ключевых внешних факторов изменений (технологии, экономика). Иллюстрация выводов графиками, диаграммами, таблицами.
**Результат:**
Формулировка основных тенденций, внешних и внутренних изменений (идет в Выводы по главе) и ссылки на источники (идут в Список литературы). -->

Для выявления ключевых тенденций развития методов и анализа особенностей  управления проектами коммерческой недвижимости в России, использован метод структурированного анализа массива источников информации. Были изучены современные научные публикации отечественных и зарубежных ученых, существующие и разрабатываемые стандарты, аналитические и статистические отчеты, обзоры и доклады в международных и национальных организаций. Обработка и визуализация полученных результатов анализа осуществлена методом построения причинно-следственных карт (casual maps), что позволило установить взаимосвязи между изучаемыми процессами и явлениями.

**Основные положения**

**Аутсорсинг**

**Инсорсинг**

**Смешанный метод**

## Анализ актуальных проблем управления коммерческой недвижимостью в России и подходов к их решению\label{sec:ch1/sect2}
\todo{
**Исходные данные:**
50+ отечественных и 30+ зарубежных научных статей за последние 5 лет;
10+ диссертаций за последние 5 лет;
15+ научных монографий без ограничения срока;
международные и российские стандарты / нормативные правовые документы / концепции / программы (при наличии)
**Содержание:**
Описание существующего научно-методического аппарата (НМА) в этой предметной области. Описание фундаментальных работ основоположников (по монографиям). Описание последующих работ (по статьям и диссертациям). Описание существующего нормативного правового поля и тенденций развития (по стандартам, концепциям, программам)
**Результат:**
Формулировка основных положений НМА в их современной редакции (идет в Выводы по главе) и ссылки на источники (идут в Список литературы).
}

**Проблемы**

**Проблемы взаимодействия с государственными органами управления и организациями-контрагентами.**

**Подходы к решению проблем управления коммерческой недвижимостью в России на основе механизмов аутсорсинга.**

В данной области исследований сформирован научный задел и имеется апробированный научно-методический аппарат, представленный работами [1–4]. Вопросы … подробно рассмотрены в работах [5, 6]. Процессы … регламентируются [7, 8]. Исследованию проблем … посвящены работы [9–11]. Результаты разработки … изложены в [12]. В работе [13] приведены… 
Методика расчета … [9, с. 75-79] позволяет определить … Методика ана-лиза … [10, с. 34-39] отличается … 

## Постановка научной задачи исследования\label{sec:ch1/sect3}
\todo{
**Исходные данные:**
Результаты параграфов 1.1 и 1.2.
**Содержание:**
Выявление противоречий между новыми потребностями практики (исходя из выводов по параграфу 1.1) и ограничениями существующего НМА (исходя из выводов по параграфу 1.2). Оценка уровня критичности противоречий, описание их последствий. Критика недостатков существующего НМА, выявление проблем его применения.
**Результат:**
Формулировка научной задачи диссертации
}

В результате анализа ... выявлены системные методологические проблемы и противоречия при ..., к которым в диссертации отнесены: .... Это обусловливает необходимость модернизации используемого научно-методического аппарата с учетом....

Анализ научных работ предшественников, выявленные методологические проблемы ..., а также обоснованные перспективы применения ..., позволили сформулировать гипотезу диссертации, поставить научную задачу и выполнить ее математическую формализацию.

**Гипотеза диссертации** состоит в том, что ...

Исходя из выдвинутой гипотезы, **научная задача диссертации** в формализованном виде состоит в том, что требуется разработать научно-методический аппарат  как совокупность образующих его методик , позволяющий на основе статистических данных ....  и прогнозных данных по ...  сформировать рациональную структуру ... . , при которой обеспечивается максимизация экономической эффективности  как функции факторов результатов  и факторов затрат, определенных на множестве вариантов возможных структур :

\todo{Излагается гипотеза исследования – как можно преодолеть противоречия? Ставится научная задача исследования – какие новые элементы НМА нужно разработать, как усовершенствовать существующий НМА. Дается математическая формализация научной задачи, обычно в виде задачи оптимизации. Дается иллюстрация структуры разрабатываемого механизма / моделей / инструментов
}

`Рисунок Логическая структура диссертационного исследования`

\section\*{Выводы по Главе 1 (результаты параграфов 1.1-1.4)}
\addcontentsline{toc}{section}{Выводы по Главе 1}  % Добавляем его в оглавление

#. Анализ тенденций развития [объекта и предмета исследования] ... позволил установить, что ....
#. В результате анализа ... выявлены системные методологические проблемы и противоречия при ..., к которым в диссертации отнесены:  ....
#. Выполненный анализ существующих научных работ в данной области исследования показал, что для ... требуется модернизация используемого научно-методического аппарата. Для этих целей обоснованы перспективы применения ... к новой области исследования. Это позволит решить методологические проблемы, вызванные ....
#. Сформулирована гипотеза диссертационного исследования, в соответствии с которой ....
#. Поставлена и математически формализована научная задача диссертационного исследования, состоящая в .... Разработана логическая структура диссертации, обеспечивающая решение поставленной научной задачи.

%**SWOT?**
<!-В данной области исследований сформирован научный задел и имеется апробированный научно-методический аппарат, представленный работами [1–4]. Вопросы ... подробно рассмотрены в работах [5, 6]. Процессы ... регламентируются [7, 8]. Исследованию проблем ... посвящены работы [9–11]. Результаты разработки ... изложены в [12]. В работе [13] приведены...
 -->

\FloatBarrier
\clearpage
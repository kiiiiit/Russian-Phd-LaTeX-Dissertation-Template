# Практическая реализация [МЕХАНИЗМА / МОДЕЛЕЙ / ИНСТРУМЕНТОВ]\label{ch:ch3}

## Формирование плана реализации [механизма / моделей / инструментов]\label{sec:ch3/sect1}
\todo{
	Описывается объект внедрения (предприятие).
	Описываются исходные данные для внедрения.
	Описывается порядок проведения эксперимента.
}

`Рисунок 3.1 показывает структуру разработанных практических [инструментов / рекомендаций]`

## Программная реализация [механизма / моделей / инструментов]\label{sec:ch3/sect2}

\todo{
	По типовой схеме ТЗ на разработку ПЭВМ.
	Описывается комплекс технических средств и пакеты прикладных программ для реализации [механизма / моделей / инструментов].
	Описываются ключевые особенности программной реализации элементов (из главы 2).
}

## Организационная реализация [механизма / моделей / инструментов]\label{sec:ch3/sect3}

\todo{
	По типовой схеме ТЗ на проект реорганизации бизнес-процессов.
	Описываются организационные изменения на предприятии (из параграфа 3.1).
}

## Апробация и оценка экономической эффективности [механизма / моделей / инструментов]\label{sec:ch3/sect4}

\todo{
	Выбираются и обосновываются показатели для оценки экономической эффективности.
	Рассчитываются и анализируются показатели по данным предприятия (из параграфа 3.1)
}

`Рисунок 3.2 содержит графики с результатами апробации и внедрения.`

\section*{Выводы по Главе 3 (результаты параграфов 3.1-3.4)}
\addcontentsline{toc}{section}{Выводы по Главе 3}  % Добавляем его в оглавление

#. Для практической реализации авторского [механизма / моделей / инструментов] в диссертации разработаны следующие практические [инструменты / рекомендации].
#. [Инструмент / рекомендация] ... реализует ....
#. [Инструмент / рекомендация] ... реализует ....
#. Апробация и внедрение [механизма / моделей / инструментов] выполнены на предприятии .... Результаты диссертации позволили обосновать управленческие решения, обеспечивающие повышение экономической эффективности...

\FloatBarrier
\clearpage
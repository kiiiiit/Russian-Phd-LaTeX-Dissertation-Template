<!--
> В заключении обычно перечисляются полученные результаты. Не надо мелочиться, если получено тридцать маленьких результатов, не пишите обо всех тридцати, пишите о трех, но пишите весомо, о том, что вы на самом деле смогли добиться. Хороший стиль, если вы в заключении пишете о нерешенных задачах, это показывает культуру автора.

> Вопрос: Как оценить практическую ценность работы и темы?
> Ответ: Я хочу сразу предостеречь всех. Практическая ценность не означает, что завтра фабрика должна начать работать с помощью ваших моделей. Самое практическое в науке это то, что вы на качественном уровне получили понимание, как устроен реальный процесс. Если вы поняли на качественном уровне природу реального процесса, это очень большое дело. Если вы создаете модели, которые сразу могут использоваться в практике, это очень хорошо. -->

#. Выполненный в диссертации анализ ... позволил выявить ключевые факторы, оказывающие влияние на ....
#. Результаты анализа ... и обзора существующих научных работ показали, что в данной области исследования имеются системные методологические проблемы, обусловленные противоречием между новыми потребностями практики ..., и ограничениями используемого научно-методического обеспечения, что свидетельствует об актуальности его совершенствования. Для его дальнейшего развития в диссертации предложено и обосновано применение к новой области исследований методов .... 
#. Выдвинута гипотеза диссертации, состоящая в том, что .... В соответствии с данной гипотезой, выполнена постановка научной задачи исследования и ее математическая формализация. Она сводится к ....
#. В целях решения поставленной научной задачи разработан научно-методический аппарат для ... в виде комплекса взаимосвязанных авторских методик:

- методики ...;
- методики ...;
- методики ....
 
Основой для разработки научно-методического аппарата послужили ..., что обусловило обоснованность и достоверность полученных результатов.
#. Для практической реализации научно-методического аппарата разработаны инструменты:

- инструмент ...;
- инструмент ...;
- инструмент ....

#. Апробация научно-методического аппарата ... и внедрение инструментов ... выполнены на предприятии .... Результаты диссертации позволили обосновать управленческие решения, обеспечивающие повышение экономической эффективности ....
#. Результаты разработки научно-методического аппарата и реализации практических инструментов позволяют сделать заключение о том, что поставленная научная задача решена, а цель диссертации достигнута. Выполненное диссертационное обеспечивает .... Направлениями дальнейших исследований по проблематике диссертации являются ....
\chapter*{Словарь терминов}             % Заголовок
\addcontentsline{toc}{chapter}{Словарь терминов}  % Добавляем его в оглавление

**Проект**, **Актив**, [**Объект коммерческой недвижимости**](https://ru.wikipedia.org/wiki/Коммерческая_недвижимость) :  это здания, сооружения или земельные участки, используемые для коммерческой деятельности с последующим извлечением постоянной прибыли или прироста капитала, дохода от аренды, инвестиционного дохода. Коммерческая недвижимость включает в себя офисные здания, объекты промышленности, гостиницы, магазины и торговые центры, сельскохозяйственные предприятия, склады, гаражи и так далее.

**УК**, [**Управляющия организация**](https://ru.wikipedia.org/wiki/Управляющая_организация) :  отличительной чертой является занятие ей только управлением объектом коммерческой недвижимости, то есть без самостоятельной эксплуатации, технического и санитарного содержания, и оказания коммунальных услуг. Для осуществления эксплуатации, технического и санитарного содержания объекта коммерческой недвижимости, а также оказания иных услуг, управляющая организация заключает соответствующие договоры.

**ЭК**, [**Эксплуатирующая организация**](https://www.consultant.ru/document/cons_doc_LAW_51040/8531126e5f632762e7d7a88d4f684fc1a153faef/) : отличительной чертой компании является выполнение ими функций эксплуатации, технического и санитарного содержания имущества по договору подряда заключенного с собственниками, либо Управляющей компанией объекта коммерческой недвижимости.

[**Аутсорсинг**](https://ru.wikipedia.org/wiki/Аутсорсинг) : это передача непрофильных функций компании сторонним организациям. Например, фирмам не всегда выгодно и удобно содержать в штате уборщиков или собственную службу безопасности. Тогда они обращаются к частным охранным предприятиям или клининговым компаниям — экономят на подборе, обучении и оформлении персонала, но закрывают нужную задачу. Слово происходит от английского *outer source using*, что можно перевести как «использование внешних источников».
%% https://vc.ru/hr/244556-pochemu-kompanii-ispolzuyut-autsorsing-na-primerah-it-gigantov
%% https://habr.com/ru/post/335206/

[**Pay-back period**](https://ru.wikipedia.org/wiki/Срок_окупаемости) : Срок окупаемости (период окупаемости) — период времени, необходимый для того, чтобы доходы, генерируемые инвестициями, покрыли затраты на инвестиции.

**Инсорсинг** : это передача проекта, задания заинтересованному работнику или отделу внутри компании без привлечения внешнего исполнителя[@Kurbatov2015].
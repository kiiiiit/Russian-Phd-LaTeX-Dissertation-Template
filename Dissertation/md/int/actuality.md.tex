%**Обоснование актуальности темы диссертации:**
В период становления Российского рынка недвижимости, с его самостоятельным путём развития, заключавшемся в оперативности трансформации бывших производственных зданий (цеха, заводоуправления, гаражи и целые троллейбусные парки и тому подобное) и торговых площадок (колхозных рынков, универсальных магазинов, стихийных торговых точек и так далее) под новый формат ведения бизнеса - многофункциональных, торговых и бизнес-центров, потребовал от собственника смены самой парадигмы отношения к новым активам - повышение капитализации через повышение стоимости самого актива, как объекта недвижимости (физическое состояние, доступность, обеспеченность ресурсами и т.д.) и повышение устойчивости финансовой модели бизнеса, возможного к ведению на территории данного конкретного объекта коммерческой недвижимости.

Процесс трансформации бизнеса - борьба за его доходность и сокращение pay-back period с целью привлечением новых собственников-акционеров, потребовал от менеджмента пересмотра подходов к обслуживанию и эксплуатации объектов. Существующая в то время практика была основана на имеющемся опыте планового хозяйства с непременным присутствием в штате организации всех обеспечивающих функций от дворника, сантехника и уборщицы до главных инженеров и прочих руководителей в полном составе, согласно разработанной государством нормативной базе, так значительно влияющих своим фондом оплаты труда на размер расходов проекта. Одним из решений проблемы сокращения расходов стало распределение прямых расходов объектов коммерческой недвижимости на несколько проектов, переведя расходы из прямых^[Под прямыми расходами понимаются затраты, непосредственно связанные с производством продукции (выполнением работ, оказанием услуг), и включаемые в себестоимость единицы учета производимой продукции (выполняемых работ, оказываемых услуг) на основании первичных учетных документов, (Инструкция по применению плана счетов бухгалтерского учета финансово-хозяйственной деятельности организаций, утвержденной приказом Минфина России от 31.10.2000 № 94)] в косвенные^[Косвенные затраты (расходы), которые организация несет в связи с одновременным производством нескольких видов продукции (работ, услуг), включаются в себестоимость каждого из них каким-либо расчетным путем, согласно выбранным организациям экономически обоснованных методов.][@Masnikov2012; @Veresagin2019], что привело к формированию  <<управляющих компаний>>, обслуживающих целые пулы, очень часто совершенно разноплановых, объектов коммерческой недвижимости одного собственника либо группы собственников.

Следующим, очевидно вытекающим из формирования управляющих компаний шагом, явился их выход на свободный и конкурентный рынок с целью дальнейшего сокращения расходов на содержание Активов и диверсификации бизнеса собственников. Тем самым мы вплотную подошли к формированию нового вида бизнеса - _Рынка аутсорсинговых услуг в сфере управления и эксплуатации коммерческой недвижимости_. Дальнейшая трансформация и становление рынка всё больше приводили к обострению нерешенного на этапе становления и всё более **актуального** стратегического вопроса: наращивать и развивать собственную ресурсную базу или заимствовать полную услугу у профильных аутсорсинговых компаний.

%\todo[inline, size=\tiny, color=gray!40]{Какие современные явления в объекте и предмете исследования обусловили необходимость их исследования? Какие проблемные ситуации возникают в практике анализа и управления? Какое значение имеют изучаемые проблемы для экономики и общества? В чем противоречие между потребностями практики и существующими научно-методическим аппаратом?}

**Содержание области исследования:** разработка экономических проблем современного состояния и прогнозирования развития строительного комплекса под влиянием таких тенденций и факторов, как реструктуризация национальных экономик, инновационные технологии, совершенствование технологической и воспроизводственной структур инвестиций, повышение роли социально-ориентированных направлений развития и др.

**Объект исследования:** строительный комплекс в целом; предприятия различных форм собственности, функционирующие в инвестиционно- строительной сфере; организационно-правовые формы взаимодействия участников инвестиционно-строительного процесса, их объединения;
государственное регулирование в сфере капитального строительства, жилищно-коммунального хозяйства, на рынке недвижимости.

**Предметом исследования** являются управленческие отношения, возникающие в процессе формирования, развития (стабилизации) и разрушения экономических систем.

**Соответствие паспорту научной специальности.** Область исследования соответствует следующим пунктам паспорта специальности 08.00.05 – Экономика и управление народным хозяйством:

- 1.3.60. Методология формирования рыночного механизма управления корпоративными структурами в строительном комплексе.
- 1.3.66. Развитие теории и методологии управления объектами недвижимости различного функционального назначения.
- 1.3.67. Теоретические и методические основы разработки и внедрения инноваций в основные, вспомогательные и обслуживающие производственные процессы по созданию, эксплуатации и обслуживанию объектов недвижимости.

**Теоретическая значимость** научной работы заключается в дальнейшем развитии научно-методического аппарата в области эксплуатации Актива и его модернизации в части разработки методик управления, с целью получения новых результатов по теме работы.

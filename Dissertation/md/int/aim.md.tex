%**Цель исследования**
является исследование причин и практики применения с последующей разработкой метода и модели оценки выбора наиболее приемлемого варианта управления конкретным Активом, обеспечивающего повышение его экономической привлекательности и эксплуатационной устойчивости.
Для достижения поставленной в диссертационной работе цели, исследователем решаются следующие основные задачи:

- проведение анализа современных экономических условий аутсорсинга в сфере коммерческой недвижимости;
- проведение анализа существующих методов расчетов и проблем их применения;
- разработка методики оценки стоимости существующей модели эксплуатации;
- разработка алгоритма расчета альтернативного сценария управления Активом;
- разработка инструмента выбора наиболее актуального для данного Актива метода управления;
- разработка рекомендаций по изменению/корректировке модели управления;
- оценка экономической эффективности предложенных изменений по результатам апробации в реальных рыночных услових на примере действующих Активов.

<!-- 
Цель диссертационного исследования - то, к чему стремится автор. Иными словами, это результат диссертации. Обычно постановка цели привязана к конкретной теме, то есть зависит от выбранного направления. В качестве цели может быть принято, к примеру, описание какого-либо нового явления, изучение и анализ этого явления и не только.

При постановке и формулировании цели диссертации обычно используются такие речевые клише: 

- «разработка нового подхода»; «установление факта»;
- «обоснование гипотезы»;
- «выявление закономерностей»;
- «определение реальной эффективности» и тому подобное.

После того, как цель была поставлена и сформулирована, нужно подготовить перечень задач, решение которых необходимо для ее достижения.

https://studwork.org/spravochnik/oformlenie/dissertaciya/celi-i-zadachi-v-dissertacionnom-issledovanii -->
<!-- Основные научные результаты, полученные в ходе исследования[@Gavrilov1988] -->
<!-- Основные положения, выносимые на защиту: -->
#. Предложена классификация методов управления коммерческой движимости учитывающая тактические и стратегические цели и задачи собственника Актива с учётом стадии жизненного цикла Актива, что позволяет обосновать принятие необходимых управленческих решений по упреждающему устранению проблемных ситуаций и определить параметры дальнейшего развития Актива.
#. Обоснован подход к институциональному управлению процессами эксплуатации Активом на основе разработанной цифровой модели, отличающейся реализацией концепции автономности принятия решения собственником для определения дальнейшей стратегии и тактики, обеспечивающий снижение транзакционных издержек взаимодействующих сторон.
#. Разработан механизм формирования независимой финансовой модели и методики определения наиболее актуального метода управления, отличающийся поэтапной алгоритмизацией и интеграцией процессов эксплуатации объектов коммерческой недвижимости при сравнении походов аутсорсинга и инсорсинга, что обеспечивает повышение финансовый показателей деятельности Актива и увеличивает срок его коммерческой эксплуатации, через развитие компетенций, необходимых именно в данные момент с учетом целей и задач.
#. Разработан механизм согласования экономических интересов участников, обеспечивающих жизнедеятельность Актива, отличающийся оптимизацией транзакционных издержек всех участников проектов,  процедурами интегральной оценки, теоретико-игровой формализацией экономических и управленческих отношений между собственниками,  менеджерами и прочими заинтересованными в успешной реализации Проекта.

<!-- 1. Предложена классификация процессов внешнеторговой деятельности промышленных предприятий, отличающаяся формализованным кодифицированным описанием стадий жизненного цикла наукоемкой промышленной продукции и каналов трансфера технологий, видов инжиниринга, объектов и стадий внешнеторговой деятельности, что позволяет выявить ключевые проблемы в данной области и обосновать подход к их решению.
2. Обоснован подход к институциональному управлению процессами внешнеторговой деятельности промышленных предприятий на основе отраслевой цифровой платформы, отличающийся реализацией концепций платформенной экономики, государственно-частного партнерства и принципа «единого окна» с учетом специфики международной промышленной кооперации, и обеспечивающий снижение транзакционных издержек взаимодействующих сторон.
3. Разработан механизм формирования состава участников и развития организационной структуры отраслевой цифровой платформы, отличающийся поэтапной алгоритмизацией и интеграцией процессов внешнеторговой деятельности промышленных предприятий с услугами органов государственного управления и коммерческими инфраструктурными организациями-контрагентами – таможенными брокерами, логистическими операторами, патентными бюро, финансово-кредитными учреждениями, страховыми, факторинговыми и лизинговыми компаниями, что обеспечивает компактификацию основных этапов осуществления экспортно-импортных операций в промышленности.
4. Разработан механизм согласования экономических интересов участников проекта создания отраслевой цифровой платформы, отличающийся оптимизацией транзакционных издержек промышленных предприятий на реализацию процессов внешнеторговой деятельности и теоретико-игровой формализацией экономических и управленческих отношений между публичным партнером, частным партнером и компанией-провайдером услуг, что позволяет обосновать экономические параметры государственно-частного партнерства при создании платформы и ее ценовую политику.
5. Разработан механизм мониторинга экспортного потенциала и внешнеторговой активности промышленных предприятий, отличающийся процедурами интегральной оценки уровня готовности предприятий, являющихся потенциальными или реальными участниками внешнеторговой деятельности, к эффективному осуществлению экспортно-импортных операций, что позволяет обосновать управленческие решения по упреждающему устранению проблемных ситуаций и определить параметры типовых договоров с компанией-провайдером отраслевой цифровой платформы.
 -->
%**Методология и методы исследования**
Информационно-эмпирическую базу исследования составили действующие нормативные правовые акты, статистические и аналитические отчеты, доклады и обзоры международных и национальных организаций, институтов, агентств и ведущих деятелей и участников отрасли, как теоретиков, так и практиков рынка.

Теоретическую, практическую, научную и методологическую основу исследования составили специальные методы теории игр, теории принятия решений, экспертного оценивания, анализа иерархий, экономического анализа и сравнений, а так же общенаучные методы, теории систем, дедукции, индукции, абстрагирования, формализации.

#. Предложен подход к оценке вариантов выбора, отличающийся от существующих эмпирических методов, и позволяющий принять взвешенное и наиболее оптимальное решение.
#. Разработана универсальная методика оценки, отличающаяся от известных методик, имеющихся у каждого из провайдеров услуг, что позволяет собственнику самостоятельно принять решение о дальнейшем методе управления.
#. Разработана финансовая модель, обеспечивающая возможность оценки влияния метода управления на капитализацию Актива и его стоимость.
#. Разработан инструмент принятия решения о выборе варианта управления, отсутствующий на рынке, и обеспечивающий соблюдение баланса интересов собственника и менеджмента Актива.
#. Разработаны рекомендации по его применению, отсутствующие на открытом рынке, и обеспечивающие корректность и объективность результатов.

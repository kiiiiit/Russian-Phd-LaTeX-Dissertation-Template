\begin{flushright}
\color{blue}
\item\textbf{Проректору}
\item ФГБОУВПО «Российская академия народного хозяйства и государственной службы при Президенте Российской Федерации»
\item\textbf{Марголину Андрею Марковичу}

\color{cyan}
\item от слушателя группы ДБА-20-1
\item Высшей школы корпоративного управления РАНХиГС
\item\textbf{Дедова Сергея  Борисовича}
\end{flushright}
\medskip

\begin{center}
\fixme{\large\textbf{Уважаемый Андрей Маркович!}}
\end{center}

Являясь активным участником учебного процесса на программе «Доктор делового администрирования» Высшей школы корпоративного управления РАНХиГС, проводя научное исследование в процессе подготовки научной работы, для получения ученой степени в рамках одного, ставшего родным, учебного заведения, прошу направить меня в Управление аспирантуры и докторантуры РАНХиГС для прикрепления с целью подготовки диссертации на соискание ученой степени кандидата экономических наук без освоения программ подготовки научно-педагогических кадров по научной специальности: 5.2.1. Экономическая теория, шифр специальности: 08.00.05 Экономика и управление народным хозяйством в рамках научных исследований Высшей школы корпоративного управления. Предполагаемая тема диссертационного исследования: «Актуальные вопросы эксплуатации коммерческой̆ недвижимости при сравнении двух подходов: аутсорсинг - внешнее управление и инсорсинг - эксплуатация собственными силами компании».

**Содержание области исследования**: разработка экономических проблем современного состояния и прогнозирования развития строительного комплекса под влиянием таких тенденций и факторов, как реструктуризация национальных экономик, инновационные технологии, совершенствование технологической и воспроизводственной структур инвестиций, повышение роли социально-ориентированных направлений развития и др.

{\aim\markdownInput{./Dissertation/md/int/aim.md}}

{\influence\markdownInput{./Dissertation/md/int/influence.md}}

Полагаю, что рассматриваемая мной тема исследования является актуальной в текущих реалиях и этапе становления рынка коммерческой недвижимости России, так как и дальнейшая трансформация и становление рынка всё больше приводят к обострению нерешенного на этапе становления стратегического вопроса: наращивать и развивать собственную ресурсную базу или заимствовать полную услугу у профильных аутсорсинговых компаний. Так как в период становления Российского рынка недвижимости, с его самостоятельным путём развития, заключавшемся в оперативности трансформации бывших производственных зданий (цеха, заводоуправления, гаражи и целые троллейбусные парки и тому подобное) и торговых площадок (колхозных рынков, универсальных магазинов, стихийных торговых точек и так далее) под новый формат ведения бизнеса - многофункциональных, торговых и бизнес-центров, потребовал от собственника смены самой парадигмы отношения к новым активам - повышение капитализации через повышение стоимости самого актива, как объекта недвижимости (физическое состояние, доступность, обеспеченность ресурсами и т.д.) и повышение устойчивости финансовой модели бизнеса, возможного к ведению на территории данного конкретного объекта коммерческой недвижимости.

Мне, как непосредственному участнику становления и развития рынка коммерческой недвижимости в России с двадцатилетним стажем и опытом, наиболее явно видны те диспропорции и отсутствие явных ориентиров, снижающих темпы развития и общие экономические показатели отрасли. Мое исследование позволит собственникам, как коммерческим, так и бюджетным организациям, принимать наиболее взвешенные решения о выборе той или иной модели развития.

Предварительная договоренность о рассмотрении диссертации с Высшей школой корпоративного управления РАНХиГС достигнута.